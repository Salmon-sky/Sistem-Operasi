\documentclass[11pt,a4paper]{article}
%%%%%%%%%%%%%%%%%%%%%%%%% Credit %%%%%%%%%%%%%%%%%%%%%%%%

% template ini dibuat oleh martin.manullang@if.itera.ac.id untuk dipergunakan oleh seluruh sivitas akademik itera.

%%%%%%%%%%%%%%%%%%%%%%%%% PACKAGE starts HERE %%%%%%%%%%%%%%%%%%%%%%%%
\usepackage{graphicx}
\usepackage{caption}
\captionsetup[table]{name=Tabel}
\captionsetup[figure]{name=Gambar}
\usepackage{tabulary}
% \usepackage{amsmath}
\usepackage{fancyhdr}
% \usepackage{amssymb}
% \usepackage{amsthm}
\usepackage{placeins}
% \usepackage{amsfonts}
\usepackage{graphicx}
\usepackage[all]{xy}
\usepackage{tikz}
\usepackage{verbatim}
\usepackage[left=2cm,right=2cm,top=3cm,bottom=2.5cm]{geometry}
\usepackage{hyperref}
\hypersetup{
    colorlinks,
    linkcolor={red!50!black},
    citecolor={blue!50!black},
    urlcolor={blue!80!black}
}
\usepackage{libertine}
\usepackage{libertinust1math}
\usepackage[T1]{fontenc}
\usepackage{inconsolata}

\usepackage{caption}
\usepackage{subcaption}
\usepackage{multirow}
\usepackage{psfrag}
\usepackage[T1]{fontenc}
\usepackage[scaled]{beramono}
% Enable inserting code into the document
\usepackage{listings}
\usepackage{xcolor} 
% custom color & style for listing
\definecolor{codegreen}{rgb}{0,0.6,0}
\definecolor{codegray}{rgb}{0.5,0.5,0.5}
\definecolor{codepurple}{rgb}{0.58,0,0.82}
\definecolor{backcolour}{rgb}{0.95,0.95,0.92}
\lstdefinestyle{mystyle}{
	backgroundcolor=\color{backcolour},   
	commentstyle=\color{green},
	keywordstyle=\color{codegreen},
	numberstyle=\tiny\color{codegray},
	stringstyle=\color{codepurple},
	basicstyle=\ttfamily\footnotesize,
	breakatwhitespace=false,         
	breaklines=true,                 
	captionpos=b,                    
	keepspaces=true,                 
	numbers=left,                    
	numbersep=5pt,                  
	showspaces=false,                
	showstringspaces=false,
	showtabs=false,                  
	tabsize=2
}
\lstset{style=mystyle}
\renewcommand{\lstlistingname}{Kode}
%%%%%%%%%%%%%%%%%%%%%%%%% PACKAGE ends HERE %%%%%%%%%%%%%%%%%%%%%%%%


%%%%%%%%%%%%%%%%%%%%%%%%% Data Diri %%%%%%%%%%%%%%%%%%%%%%%%
\newcommand{\stuid}{120140233}
\newcommand{\student}{\textbf{Muhammad Khadziq (\stuid{})}}
\newcommand{\course}{\textbf{Sistem Operasi RD (IF2223)}}
\newcommand{\assignment}{\textbf{2}} % tugas ke...

%%%%%%%%%%%%%%%%%%% using theorem style %%%%%%%%%%%%%%%%%%%%
\newtheorem{thm}{Theorem}
\newtheorem{lem}[thm]{Lemma}
\newtheorem{defn}[thm]{Definition}
\newtheorem{exa}[thm]{Example}
\newtheorem{rem}[thm]{Remark}
\newtheorem{coro}[thm]{Corollary}
\newtheorem{quest}{Question}[section]
%%%%%%%%%%%%%%%%%%%%%%%%%%%%%%%%%%%%%%%%
\usepackage{lipsum}%% a garbage package you don't need except to create examples.
\usepackage{fancyhdr}
\usepackage[ddmmyyyy]{datetime}
\pagestyle{fancy}
\lhead{Muhammad Khadziq (120140233)}
\rhead{ \thepage}
\cfoot{\textbf{HandsOn 2 : Synchronisation and Deadlock}} % ini untuk judul tugas
\renewcommand{\headrulewidth}{0.4pt}
\renewcommand{\footrulewidth}{0.4pt}

%%%%%%%%%%%%%%  Shortcut for usual set of numbers  %%%%%%%%%%%

\newcommand{\N}{\mathbb{N}}
\newcommand{\Z}{\mathbb{Z}}
\newcommand{\Q}{\mathbb{Q}}
\newcommand{\R}{\mathbb{R}}
\newcommand{\C}{\mathbb{C}}
\setlength\headheight{14pt}

%%%%%%%%%%%%%%%%%%%%%%%%%%%%%%%%%%%%%%%%%%%%%%%%%%%%%%%555

\begin{document}
\thispagestyle{empty}
\begin{center}
	\includegraphics[scale = 0.15]{Figure/ifitera-header.png}
	\vspace{0.1cm}
\end{center}
\noindent
% change font family for header section only
%{\fontfamily{LinuxLibertineT-OsF}\large\selectfont 
{\large
\rule{17cm}{0.2cm}\\[0.3cm]
Nama: \student \hfill Tugas Ke: \assignment\\[0.1cm]
Mata Kuliah: \course \hfill Tanggal: \today\\
\rule{17cm}{0.05cm}
\vspace{0.1cm}
}


%%%%%%%%%%%%%%%%%%%%%%%%%%%%%%%%%%%%%%%%%%%%% BODY DOCUMENT %%%%%%%%%%%%%%%%%%%%%%%%%%%%%%%%%%%%%%%%%%%%%
\section{Tujuan HandsOn}
    Tujuan saya melakukan Hands On  kedua adalah untuk memahami bagaimana sistem disinkronkan dan masalah apa yang ada, dan untuk memahami solusi saat menjalankan bagian penting. Ada beberapa implementasi yang harus disertakan dalam Hands On kedua ini, antara lain: join menggunakan Semaphores, Binary Semaphores, Produces Consumer, Reader/Writer Locks, dan Dining Philosophers.
    
\section{Join}
\subsection{Code Snippets}
    \begin{lstlisting}[language=C, caption=Captionnya tulis di sini class,label={labelkode}]
    class SynthiaDataset(Dataset):
    #include <stdio.h>
    #include <stdlib.h>
    #include <pthread.h>
    #include <unistd.h>

    #include "common.h"
    #include "common_threads.h"

    #ifdef linux
    #include <semaphore.h>
    #elif APPLE
    #include "zemaphore.h"
    #endif

    sem_t s;

    void *child(void *arg) {
        sleep(2);
        printf("child\n");
        Sem_post(&s); // signal here: child is done
        return NULL;
    }

    int main(int argc, char *argv[]) {
        Sem_init(&s, 0); 
        printf("parent: begin\n");
        pthread_t c;
        Pthread_create(&c, NULL, child, NULL);
        Sem_wait(&s); // wait here for child
        printf("parent: end\n");
        return 0;
    }
    \end{lstlisting}
\subsection{Gambar}
    Semaphore adalah  struktur data komputer yang sangat berguna untuk menyinkronkan proses 
 dalam  program kontrol untuk menjalankan proses. Contohnya adalah ketika utas 
 menunggu daftar untuk daftar kosong atau  kosong. Dari kondisi ini, semaphore  
 akan didefinisikan dan diinisialisasi ke nol oleh Sem init. Tujuan dari proses ini adalah semaphore akan dibagikan di antara utas dari proses yang sama. Kemudian, ketika pembuatan utas selesai, ia akan terus memanggil fungsi  semaphore child, yang akan memberi sinyal bahwa proses child selesai.

    
\begin{figure}[h]
    \centering
    \includegraphics[scale = 0.9]{Figure/join.png}
    \caption{join}
    \label{fig:join}
\end{figure}
\section{Binary}
\subsection{Code Snippets}
    \begin{lstlisting}[language=C, caption=Captionnya tulis di sini class,label={labelkode}]
    class SynthiaDataset(Dataset):
    #include <stdio.h>
    #include <stdlib.h>
    #include <pthread.h>
    #include <unistd.h>

    #include "common.h"
    #include "common_threads.h"

    #ifdef linux
    #include <semaphore.h>
    #elif APPLE
    #include "zemaphore.h"
    #endif

    sem_t mutex;
    volatile int counter = 0;

    void *child(void *arg) {
        int i;
        for (i = 0; i < 10000000; i++) {
	    Sem_wait(&mutex);
	    counter++;
	    Sem_post(&mutex);
        }
        return NULL;
    }

    int main(int argc, char *argv[]) {
        Sem_init(&mutex, 1); 
        pthread_t c1, c2;
        Pthread_create(&c1, NULL, child, NULL);
        Pthread_create(&c2, NULL, child, NULL);
        Pthread_join(c1, NULL);
        Pthread_join(c2, NULL);
        printf("result: %d (should be 20000000)\n", counter);
        return 0;
    }
    \end{lstlisting}
\subsection{Gambar}
    Command sed digunakan untuk menampilkan teks dari sebuah file agar dihapus beberapa bagiannya, bisa dihapus depan, belakang atau depan dan belakangnya. Namun tidak mempengaruhi isi file tersebut.
    \begin{figure}[h]
        \centering
        \includegraphics[scale = 0.9]{Figure/binary.png}
        \caption{Binary}
        \label{fig:tut2_1}
    \end{figure}
\section{Producer Consumer Works}
\subsection{Code Snippets}
    Berikut ini adalah contoh dari penggunaan $\backslash${\tt{begin{lstlisting}}} untuk menulis potongan kode. Dalam kasus ini saya menggunakan bahasa Python. Jika anda menggunakan C atau yang lainnya, tinggal sesuaikan di bagian parameter dari $\backslash${\tt{begin{lstlisting}}}. Anda dapat melihatnya pada code snipptes \ref{labelkode}
    
    \begin{lstlisting}[language=Python, caption=Captionnya tulis di sini class,label={labelkode}]
    
    class SynthiaDataset(Dataset):
    #include <stdio.h>
    #include <unistd.h>
    #include <assert.h>
    #include <pthread.h>
    #include <stdlib.h>

    #include "common.h"
    #include "common_threads.h"

    #ifdef linux
    #include <semaphore.h>
    #elif APPLE
    #include "zemaphore.h"
    #endif

    int max;
    int loops;
    int *buffer;

    int use  = 0;
    int fill = 0;

    sem_t empty;
    sem_t full;
    sem_t mutex;

    #define CMAX (10)
    int consumers = 1;

    void do_fill(int value) {
        buffer[fill] = value;
        fill++;
        if (fill == max)
    	fill = 0;
    }

    int do_get() {
        int tmp = buffer[use];
        use++;
       if (use == max)
    	use = 0;
        return tmp;
    }

    void *producer(void *arg) {
        int i;
        for (i = 0; i < loops; i++) {
    	Sem_wait(&empty);
    	Sem_wait(&mutex);
    	do_fill(i);
    	Sem_post(&mutex);
    	Sem_post(&full);
        }

        // end case
        for (i = 0; i < consumers; i++) {
	    Sem_wait(&empty);
	    Sem_wait(&mutex);
	    do_fill(-1);
	    Sem_post(&mutex);
	    Sem_post(&full);
        }

        return NULL;
    }
                                                                               
    void *consumer(void *arg) {
        int tmp = 0;
        while (tmp != -1) {
	    Sem_wait(&full);
	    Sem_wait(&mutex);
	    tmp = do_get();
	    Sem_post(&mutex);
	    Sem_post(&empty);
	    printf("%lld %d\n", (long long int) arg, tmp);
        }
        return NULL;
    }

    int main(int argc, char *argv[]) {
        if (argc != 4) {
	    fprintf(stderr, "usage: %s <buffersize> <loops> <consumers>\n", argv[0]);
	    exit(1);
        }
        max   = atoi(argv[1]);
        loops = atoi(argv[2]);
        consumers = atoi(argv[3]);
        assert(consumers <= CMAX);

        buffer = (int *) malloc(max * sizeof(int));
        assert(buffer != NULL);
        int i;
        for (i = 0; i < max; i++) {
	    buffer[i] = 0;
        }

        Sem_init(&empty, max); // max are empty 
        Sem_init(&full, 0);    // 0 are full
        Sem_init(&mutex, 1);   // mutex

        pthread_t pid, cid[CMAX];
        Pthread_create(&pid, NULL, producer, NULL); 
        for (i = 0; i < consumers; i++) {
	    Pthread_create(&cid[i], NULL, consumer, (void *) (long long int) i); 
        }
        Pthread_join(pid, NULL); 
        for (i = 0; i < consumers; i++) {
	    Pthread_join(cid[i], NULL); 
        }
        return 0;
    }
    \end{lstlisting}
\subsection{Gambar}
    Pada program implementasi Producer/Consumer ini bisa kita sebut dengan bounded buffer. Isi program tersebut ialah memanggil, mengurangi, menghalangi konsumer, dan menunggu thread lain agar dapat memanggil
    Sem post saat terjadi full. Selanjutnya, program akan memulai fungsi procedure yang berguna dalam memanggil Sem wait(empty) dan Sem post(mutex). Pada fungsi procedur juga menjalankan terus sampai empty tadi
    menjadi max. Producer akan melakukan pengisian dengan fungsi do fill di entry pertama buffer setelah empty
    berkurang hingga mencapai nilai 0. Berikutnya, producer akan terus berjalan sampai suatu saat nanti memanggil Sem post(mutex) dan Sem post(full) yang mana akan mengganti nilai value full dari nilai -1 menjadi
    0. Sehingga, Consumer akan melakukan fungsi looping ulang dan memblok dengan value empty semaphore
    bernilai kosong.

    \begin{figure}[h]
        \centering
        \includegraphics[scale = 0.6]{Figure/producer.png}
        \caption{Shell nano command}
        \label{fig:tut3_1}
    \end{figure}
\section{Reader or Writer Locks}
\subsection{Code Snippets}
    \begin{lstlisting}[language=C, caption=Captionnya tulis di sini class,label={labelkode}]
    
    class SynthiaDataset(Dataset):

    #include <stdio.h>
    #include <stdlib.h>
    #include <pthread.h>
    #include <unistd.h>

    #include "common.h"
    #include "common_threads.h"

    #ifdef linux
    #include <semaphore.h>
    #elif APPLE
    #include "zemaphore.h"
    #endif

    typedef struct _rwlock_t {
        sem_t writelock;
        sem_t lock;
        int readers;
    } rwlock_t;

    void rwlock_init(rwlock_t *lock) {
        lock->readers = 0;
        Sem_init(&lock->lock, 1); 
        Sem_init(&lock->writelock, 1); 
    }

    void rwlock_acquire_readlock(rwlock_t *lock) {
        Sem_wait(&lock->lock);
        lock->readers++;
        if (lock->readers == 1)
	    Sem_wait(&lock->writelock);
        Sem_post(&lock->lock);
    }

    void rwlock_release_readlock(rwlock_t *lock) {
        Sem_wait(&lock->lock);
        lock->readers--;
        if (lock->readers == 0)
	    Sem_post(&lock->writelock);
        Sem_post(&lock->lock);
    }

    void rwlock_acquire_writelock(rwlock_t *lock) {
        Sem_wait(&lock->writelock);
    }

    void rwlock_release_writelock(rwlock_t *lock) {
        Sem_post(&lock->writelock);
    }

    int read_loops;
    int write_loops;
    int counter = 0;

    rwlock_t mutex;

    void *reader(void *arg) {
        int i;
        int local = 0;
        for (i = 0; i < read_loops; i++) {
	    rwlock_acquire_readlock(&mutex);
	    local = counter;
	    rwlock_release_readlock(&mutex);
	    printf("read %d\n", local);
        }
        printf("read done: %d\n", local);
        return NULL;
    }

    void *writer(void *arg) {
        int i;
        for (i = 0; i < write_loops; i++) {
	    rwlock_acquire_writelock(&mutex);
	    counter++;
	    rwlock_release_writelock(&mutex);
        }
        printf("write done\n");
        return NULL;
    }

    int main(int argc, char *argv[]) {
        if (argc != 3) {
	    fprintf(stderr, "usage: rwlock readloops writeloops\n");
	    exit(1);
        }
        read_loops = atoi(argv[1]);
        write_loops = atoi(argv[2]);
    
        rwlock_init(&mutex); 
        pthread_t c1, c2;
        Pthread_create(&c1, NULL, reader, NULL);
        Pthread_create(&c2, NULL, writer, NULL);
        Pthread_join(c1, NULL);
        Pthread_join(c2, NULL);
        printf("all done\n");
        return 0;
    }
    \end{lstlisting}
\subsection{Gambar}
    Pada program implementasi Writer Locks, dapat dilihat bahwa terdapat classic problem dari flexible
    locking primitive yang menunjukkan bahwa akses struktur data berbeda perlu dibutuhkannya kunci spesial
    yang dibuat sebagai pembantu tipe operasi seperti reader/writer locks. Jika suatu saat thread memperbarui
    struktur datanya agar dapat memanggil pasangan operasi sinkronisasi rwlock acquire writelock yang mana
    berfungsi dalam mendapatkan writelock dan rwlock release writelock untuk melepaskannya. Secara umumnya,
    semaphore writelock untuk memastikan hanya satu writer saja yang mendapatkan lock dan memperbarui
    struktur datanya dengan masuknya ke critical section.
    
    \begin{figure}[h]
        \centering
        \includegraphics[scale = 0.6]{Figure/rwlock.png}
        \caption{Membuat shell script di nano}
        \label{fig:asg6_1}
    \end{figure}
\section{Dining Philosophers}
\subsection{Deadlock}
\subsubsection{Code Snippets}
    Berikut ini adalah contoh dari penggunaan $\backslash${\tt{begin{lstlisting}}} untuk menulis potongan kode. Dalam kasus ini saya menggunakan bahasa Python. Jika anda menggunakan C atau yang lainnya, tinggal sesuaikan di bagian parameter dari $\backslash${\tt{begin{lstlisting}}}. Anda dapat melihatnya pada code snipptes \ref{labelkode}
    
    \begin{lstlisting}[language=C, caption=Captionnya tulis di sini class,label={labelkode}]
    
    class SynthiaDataset(Dataset):

    #include <stdio.h>
    #include <stdlib.h>
    #include <pthread.h>

    #include "common.h"
    #include "common_threads.h"

    #ifdef linux
    #include <semaphore.h>
    #elif APPLE
    #include "zemaphore.h"
    #endif

    typedef struct {
        int num_loops;
        int thread_id;
    } arg_t;

    sem_t forks[5];
    sem_t print_lock;

    void space(int s) {
        Sem_wait(&print_lock);
        int i;
        for (i = 0; i < s * 10; i++)
	    printf(" ");
    }

    void space_end() {
        Sem_post(&print_lock);
    }

    int left(int p)  {
        return p;
    }

    int right(int p) {
        return (p + 1) % 5;
    }

    void get_forks(int p) {
        space(p); printf("%d: try %d\n", p, left(p)); space_end();
        Sem_wait(&forks[left(p)]);
        space(p); printf("%d: try %d\n", p, right(p)); space_end();
        Sem_wait(&forks[right(p)]);
    }

    void put_forks(int p) {
        Sem_post(&forks[left(p)]);
        Sem_post(&forks[right(p)]);
    }

    void think() {
        return;
    }

    void eat() {
        return;
    }

    void *philosopher(void *arg) {
        arg_t *args = (arg_t *) arg;

        space(args->thread_id); printf("%d: start\n", args->thread_id); space_end();

        int i;
        for (i = 0; i < args->num_loops; i++) {
	    space(args->thread_id); printf("%d: think\n", args->thread_id); space_end();
	    think();
	    get_forks(args->thread_id);
	    space(args->thread_id); printf("%d: eat\n", args->thread_id); space_end();
	    eat();
	    put_forks(args->thread_id);
	    space(args->thread_id); printf("%d: done\n", args->thread_id); space_end();
        }
        return NULL;
    }
                                                                             
    int main(int argc, char *argv[]) {
        if (argc != 2) {
	    fprintf(stderr, "usage: dining_philosophers <num_loops>\n");
	    exit(1);
        }
        printf("dining: started\n");
    
        int i;
        for (i = 0; i < 5; i++) 
	    Sem_init(&forks[i], 1);
        Sem_init(&print_lock, 1);

        pthread_t p[5];
        arg_t a[5];
        for (i = 0; i < 5; i++) {
	    a[i].num_loops = atoi(argv[1]);
	    a[i].thread_id = i;
	    Pthread_create(&p[i], NULL, philosopher, &a[i]);
        }

        for (i = 0; i < 5; i++) 
	    Pthread_join(p[i], NULL); 

        printf("dining: finished\n");
        return 0;
    }
    \end{lstlisting}
\subsubsection{Gambar}
    \begin{figure}[h]
        \centering
        \includegraphics[scale = 0.2]{Figure/dining deadlock.png}
        \caption{Membuat teks dengan format txt}
        \label{fig:asg8_1}
    \end{figure}
    \newline
    Scripting code bash di shell menggunakan nano open editor.
    \begin{figure}[h]
        \centering
        \includegraphics[scale = 0.2]{Figure/dining deadlock(2).png}
        \caption{Shell scripting}
        \label{fig:asg8_2}
    \end{figure}
\subsection{No Deadlock}
\subsubsection{Code Snippets}
    \begin{lstlisting}[language=C, caption=Captionnya tulis di sini class,label={labelkode}]
    
    class SynthiaDataset(Dataset):

    #include <stdio.h>
    #include <stdlib.h>
    #include <pthread.h>

    #include "common.h"
    #include "common_threads.h"

    #ifdef linux
    #include <semaphore.h>
    #elif APPLE
    #include "zemaphore.h"
    #endif

    typedef struct {
        int num_loops;
        int thread_id;
    } arg_t;

    sem_t forks[5];
    sem_t print_lock;

    void space(int s) {
        Sem_wait(&print_lock);
        int i;
        for (i = 0; i < s * 10; i++)
	    printf(" ");
    }

    void space_end() {
        Sem_post(&print_lock);
    }

    int left(int p)  {
        return p;
    }

    int right(int p) {
        return (p + 1) % 5;
    }

    void get_forks(int p) {
        if (p == 4) {
	    space(p); printf("4 try %d\n", right(p)); space_end();
	    Sem_wait(&forks[right(p)]);
	    space(p); printf("4 try %d\n", left(p)); space_end();
	    Sem_wait(&forks[left(p)]);
        } else {
	    space(p); printf("try %d\n", left(p)); space_end();
	    Sem_wait(&forks[left(p)]);
	    space(p); printf("try %d\n", right(p)); space_end();
	    Sem_wait(&forks[right(p)]);
        }
    }

    void put_forks(int p) {
        Sem_post(&forks[left(p)]);
        Sem_post(&forks[right(p)]);
    }

    void think() {
        return;
    }

    void eat() {
        return;
    }

    void *philosopher(void *arg) {
        arg_t *args = (arg_t *) arg;

        space(args->thread_id); printf("%d: start\n", args->thread_id); space_end();

        int i;
        for (i = 0; i < args->num_loops; i++) {
	    space(args->thread_id); printf("%d: think\n", args->thread_id); space_end();
	    think();
	    get_forks(args->thread_id);
	    space(args->thread_id); printf("%d: eat\n", args->thread_id); space_end();
	    eat();
	    put_forks(args->thread_id);
	    space(args->thread_id); printf("%d: done\n", args->thread_id); space_end();
        }
        return NULL;
    }
                                                                             
    int main(int argc, char *argv[]) {
        if (argc != 2) {
	    fprintf(stderr, "usage: dining_philosophers <num_loops>\n");
	    exit(1);
        }
        printf("dining: started\n");
    
        int i;
        for (i = 0; i < 5; i++) 
	    Sem_init(&forks[i], 1);
        Sem_init(&print_lock, 1);

        pthread_t p[5];
        arg_t a[5];
        for (i = 0; i < 5; i++) {
	    a[i].num_loops = atoi(argv[1]);
	    a[i].thread_id = i;
	    Pthread_create(&p[i], NULL, philosopher, &a[i]);
        }

        for (i = 0; i < 5; i++) 
	    Pthread_join(p[i], NULL); 

        printf("dining: finished\n");
        return 0;
    }
    \end{lstlisting}
\subsubsection{Gambar}
    \begin{figure}[h]
	\centering
	\begin{subfigure}[b]{0.3\textwidth}
		\centering
		\def\svgwidth{\columnwidth}
		\includegraphics[width=1\textwidth]{Figure/dining no deadlock.png}
		\caption{Augment Result 1}
		\label{fig:aug-1}
	\end{subfigure}
	\qquad %add desired spacing between images, e. g. ~, \quad, \qquad, \hfill etc. 
	%(or a blank line to force the subfigure onto a new line)
	\begin{subfigure}[b]{0.2\textwidth}
		\centering
		\def\svgwidth{\columnwidth}
		\includegraphics[width=1\textwidth]{Figure/dining no deadlock(2).png}
		\caption{Augment Result 2}
		\label{fig:aug-2}
	\end{subfigure}
	\caption{Augmentation Samples}\label{fig:aug}
\end{figure}
\qquad %add desired spacing between images, e. g. ~, \quad, \qquad, \hfill etc. 
	%(or a blank line to force the subfigure onto a new line)
	\begin{subfigure}[b]{0.3\textwidth}
		\centering
		\def\svgwidth{\columnwidth}
		\includegraphics[width=1\textwidth]{Figure/dining no deadlock(3).png}
		\caption{Augment Result 2}
		\label{fig:aug-2}
	\end{subfigure}
	\caption{Augmentation Samples}\label{fig:aug}
\end{figure}

\section{Kesimpulan}
    Menggunakan terminal linux secara langsung untuk pertama kali memang banyak kesulitan, namun ketika sudah terbiasa menjalankan program atau mengubah sebuah program lebih \textit{powerful} atau lebih bebas dengan linux karena aksesnya yang luas. Kemudian banyaknya perintah di terminal linux tidak mungkin untuk dihafalkan, tapi jika sering diaplikasikan akan menjadi terbiasa dan tidak perlu lagi melihat dokumentasi.
    Dan yang saya dapatkan setelah mengerjakannya ialah saya dapat mengerti lebih dalam
    dengan materi Synchronisation and Deadlock dengan adanya pemberian kode program dari suatu programmer
    yang membuat programnya sesuai dengan implementasi materi tersebut. Dan begitu juga saya mengerti
    tentang adanya penggunaan Semaphore pada program yang telah dijalankan. Dengan demikian, saya dapat
    menjelajahi lebih dalam terkait dengan materi sinkronisasi dan deadlock ini atas tugas Hands On 2 yang telah
    diberikan.
\section{Link Google Drive}
    \href{https://drive.google.com/drive/folders/1iAnvNA6L2OmfjI7Oc32XEHpBvSAv3ldh}{Klik di sini}
\end{document}